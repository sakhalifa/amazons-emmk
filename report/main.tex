\documentclass{article}

\author{Youri Chancrin - Théo Descomps - Samuel Khalifa - Jean-Baptiste Martinez}
\title{Rapport de projet Amazons}

\begin{document}
\maketitle

\section{Présentation du sujet}

Points requis :
- Modéliser le jeu de plateau Amazons
=> Besoin d'un plateau de jeu
- Structure Client-Serveur
Cela implique : Isolation des variables

- Utilisation de GSL et de graphes

- Utilisation de dllib
Cela implique : Création de librairie dynamiques

\section{Squelette de résolution}
On découpe le projet en 3 parties
- Serveur
- Client
- Common

Boucle de jeu : (à décrire)

Le serveur s'occupe de gérer une partie jouée par 2 clients. 
Le module Common contient un ensemble de fonctionnalités qui sont 
utilisées la fois par le serveur et par les clients. On y retrouve 
notamment diverses structures de données :
- arrayList
- Ensemble de positions
- Arbre

Ces structures offrent une diversité de complexité dont les clients 
peuvent se servir pour minimiser leurs temps d'éxecutions.

\section{Isolation de la mémoire}
=> Le serveur copie des données et les mets a disposition de chacun
des clients, dans le cas ou les clients et le serveur seraient sur 
des machines séparées, les clients devraient allouer leur mémoire.

\section{Immutabilité du graphe}
=> Le graphe est sous un format optimiser, on choisit de ne jamais 
le modifier car c'est une opération couteuse et complexe. Pour 
enregistrer les modifications au cours de la partie, on utilise des 
tableaux représentant l'ensemble des cases du plateau.

\section{IA du client}
samuel la c'est pour toi

\section{Optimisations}
Modifications du code après profilage pour réduire le temps 
d'execution des clients.
- Optimisation des calculs sur plateau en grille.


\section{Conclusion}
On é trop fort
\end{document}
