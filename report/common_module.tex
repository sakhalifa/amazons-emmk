Le module Common contient un ensemble de fonctionnalités qui sont 
utilisées à la fois par le serveur et par les clients. 

\subsection{Structures de données}
Le module Common dispose de plusieurs structures de données et opérateurs utils :
\begin{itemize}
    \item \textbf{ArrayList} : Une liste d'objet quelconque.
    \item \textbf{PositionSet} : Un set de positions.
    \item \textbf{Arbre} : Un arbre d'objet quelconque. Utilise ArrayList.
    \item \textbf{Hash de Zobrist} : Permet de transformer un plateau de jeu en hash.
    \item \textbf{Board} : Une représentation du plateu de jeu.
\end{itemize}

Ces structures offrent une diversité de complexités dont les clients 
peuvent se servir pour minimiser leurs temps d'éxecutions.

Les avoir à disposition de tous les clients simplifie le prototypage de nouvelles stratégies.
Par exemple, la structure Arbre permet de modéliser un arbre des parties possibles pour les 
stratégies qui gardent en mémoire une prévision des prochains tours.

\subsection{Plateau de jeu}
Le plateau de jeu enveloppe le graphe fournit par GSL.
Ce-dernier étant dans un format optimisé, on choisit de ne jamais 
le modifier car c'est une opération couteuse et complexe. Pour 
enregistrer les évolutions de la partie, on utilise des 
tableaux représentant l'ensemble des cases du plateau.
Ces tableaux contiennent les flêches ainsi que les reines 
et permettent d'accéder au contenu d'une case en temps constant.