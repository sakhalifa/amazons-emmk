Le module \verb|common| contient un ensemble de fonctionnalités qui sont 
utilisées à la fois par le serveur et par les clients. 

\subsection{Structures de données}
Le module \verb|common| dispose de plusieurs structures de données:
\begin{itemize}
    \item \textbf{array\_list} : Un tableau dynamique pouvant contenir n'importe quel objet.
    \item \textbf{position\_set} : Un ensemble de positions.
    \item \textbf{tree} : Un arbre d'objets quelconque.
    \item \textbf{zobrist\_hash} : Permet de \textit{hasher}\footnote{Un hash consiste à calculer un nombre le plus unique possible d'une structure de données quelconque} un plateau de jeu.
    \item \textbf{board} : Une représentation du plateu de jeu.
\end{itemize}

Ces structures offrent une diversité de complexités dont les clients 
peuvent se servir pour minimiser leurs temps d'éxecutions.

Les avoir à disposition de tous les clients simplifie le prototypage de nouvelles stratégies.
Par exemple, la structure \verb|tree| permet de modéliser un arbre des parties possibles pour les 
stratégies qui gardent en mémoire une prévision des prochains tours.

\subsection{Plateau de jeu}
Le plateau de jeu enveloppe le graphe fournit par GSL.
Ce-dernier étant dans un format optimisé, on a choisit de ne jamais 
le modifier car c'est une opération couteuse et complexe. Pour 
enregistrer les évolutions de la partie, nous avons décidé de stocker les
flèches posées sur la plateau comme un booléen indiquant si la case $c$ est bloquée par une flèche ou non.
Pour les reines, nous avons décidé de d'abord stocker la position des reines pour chaque joueur,
puis après quelques étapes de \emph{profilage} pour optimiser nos clients, nous avons rajouté
un autre moyen d'accéder aux reines sur le plateau.