L'objectif de ce projet était la création d'un serveur\footnote{Un serveur au sens figuré du terme. Nous n'avons pas eu a faire de réseau, heureusement.}
sur lequel des joueurs (\textit{clients}) peuvent jouer au jeu des \textit{Amazones}. Il nous a aussi été demandé d'implémenter quelques clients même si cela était plutôt secondaire.
Le tout en utilisant le langage de programmation \emph{C}.

\subsection{Le jeu des Amazones}

Le jeu des amazons est un jeu de plateau où deux joueurs s'affrontent en jouant tour à tour.
Ils disposent chacun de reines\footnote{Plus exactement $\mathtt{floor}(4*(m/10 + 1))$ où $m$ est la largeur du plateau},
qui peuvent se déplacer comme des reines au jeu des échecs. Les reines doivent être espacés d'au moins 2 espaces ou d'un espace en diagonale.
Le jeu requiert que les reines soient symmétriques entre les joueurs et qu'il y ait autant de reines sur la ligne externe que sur les colonnes externes.

\begin{figure}[h!]
	\centering
	\newchessgame[
		setwhite={qb1, qg1, qa2, qh2},
		addblack={qb8, qg8, qa7, qh7}
	]
	\chessboard[showmover=false]
	\caption{Un jeu des Amazones.}
	\label{fig:amazon-game}
\end{figure}

Chaque joueur joue une reine puis tire une flêche depuis la position atteinte par la reine.
Les flêches agissent comme une reine dans le jeu des échecs mais ne peuvent traverser de reines ou d'autres flêches.
Les reines ne peuvent pas se déplacer sur des flèches ou d'autres reines.
Le premier joueur qui ne peut pas jouer un coup valide perd.
\medbreak
Pour répondre à ce besoin, nous avons suivis les spécifications du sujet,
mis en place une infrastructure client-serveur, et réalisé plusieurs clients 
capable de jouer une partie, et souvent de la gagner.

\subsection{Spécifications du sujet}
Le sujet dispose de deux grands types de spécifications.
Tout d'abord, une spécification structurelle, le client-serveur.
Secondement, des spécifications d'implémentations, avec l'Utilisation 
de GSL et de dlopen.

\subsubsection{Structure Client Serveur}

Le projet doit contenir un executable actant comme un serveur responsable d'une partie entre deux joueurs.
Le serveur doit s'assurer du lancement et du déroulement de la partie. Une fois celle-ci finie, 
il doit déclarer le vainqueur.

Le serveur est responsable de l'isolation des variables. 
C'est à lui de faire des copies du plateau du jeu au début de la partie et de les fournir aux clients 
pour qu'ils s'initialisent correctement.

Les clients doivent respecter une interface afin de garantir qu'ils puissent tous s'affronter,
quelque soit le serveur qui héberge la partie.

\subsubsection{Utilisation de GSL}

Pour représenter le plateau de jeu, la GNU Scientific Library (GSL) est utilisée.

Quatres plateaux de jeux doivent être implémentés, un carré, un donut, un huit et un trèfle.

\subsubsection{Utilisation de dllib}

Les clients sont des librairies dynamiques chargées avant le début de la partie par le serveur.
Le chargement doit être réalisé avec dllib.