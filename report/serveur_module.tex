Le serveur s'occupe de gérer une partie jouée par deux clients. 

D'une part, il s'occupe de l'isolation de la mémoire lors de l'initialisation.
D'autre part, il doit être capable de déterminer le gagnant de la partie.


\subsection{Isolation de la mémoire}
Il nous a été nécessaire d'isoler la mémoire entre chaque clients.
En effet, le C donnant un accès complet à la mémoire, et vu que le serveur et les clients
sont dans le même processus, il est extrêmement risqué de donner le même pointeur au plateau à chaque client.
Pour éviter cela, le serveur doit copier les données d'initialisation de la partie pour chaque client.

Dans le cas où les clients et le serveur seraient sur des machines séparées, 
la question ne se poserait pas et les clients seraient contraints
d'allouer eux-mêmes la mémoire pour cette tâche.

\subsection{Détermination du vainqueur}
Au fur et à mesure de la partie, le serveur reçoit les coups des deux joueurs.
Il applique les modifications engendrées par ces coups sur un plateau dont seul lui a l'accès.
Ce plateau est fourni par le module \verb|common|. Lorsque que le serveur obtient le coup 
du joueur courrant, il effectue un test à l'aide la fonction \textbf{is\_move\_legal()} de l'interface board.h.
S'il reçoit la valeur false, le joueur courrant à perdu, l'autre joueur est donc le vainqueur.