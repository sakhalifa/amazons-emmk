\subsection{Spécifications du sujet}
Le sujet dispose de deux grands types de spécifications.
Tout d'abord, une spécification structurelle, le client-serveur.
Secondement, des spécifications d'implémentations, avec l'Utilisation 
de GSL et de dlopen.

\subsubsection{Structure Client Serveur}

Le projet doit contenir un executable actant comme un serveur responsable d'une partie entre deux joueurs.
Le serveur doit s'assurer du lancement et du déroulement de la partie. Une fois celle-ci finie, 
il doit déclarer le vainqueur.

Le serveur est responsable de l'isolation des variables. 
C'est à lui de faire des copies du plateau du jeu au début de la partie et de les fournir aux clients 
pour qu'ils s'initialisent correctement.

Les clients doivent respecter une interface afin de garantir qu'ils puissent tous s'affronter,
quelque soit le serveur qui héberge la partie.

\subsubsection{Utilisation de GSL}

Pour représenter le plateau de jeu, la GNU Scientific Library (GSL) est utilisée.

Quatres plateaux de jeux doivent être implémentés, un carré, un donut, un huit et un trèfle.

\subsubsection{Utilisation de dllib}

Les clients sont des librairies dynamiques chargées avant le début de la partie par le serveur.
Le chargement doit être réalisé avec dllib.