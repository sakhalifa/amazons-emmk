\subsection{Spécifications du sujet}
Le sujet nous demandait d'implémenter un modèle client-serveur et nous a
contraint à utiliser la librairie GSL.

\subsubsection{Structure Client Serveur}

La structure client-serveur, autrement appelée "monolithe", permet de séparer
la responsabilité des joueurs et du jeu en lui même. Le serveur représente le jeu. Il organise le
déroulement du jeu, retransmet l'information d'un client $x$ aux autres clients et vérifie que tout coup joué est valide.

\subsubsection{Utilisation de GSL}

Il nous a été demandé de représenter le plateau de jeu comme un graphe. 
Cela permet quelques folies pour implémenter moultes plateaux intéressant.
Dans les règles des échecs, une case aura au plus 8 voisins, cela veut dire que dans le graphe, un noeud aura au maximum 8 voisins, alors qu'un graphe fait au moins 25 noeuds.
Pour représenter ce graphe, il nous a été demandé de le représenter par sa matrice d'adjacence. 
Pour stocker cette matrice, nous étions contraints d'utiliser la GNU Scientific Library (GSL).
Cette librairie permet le stockage et traitement de matrices optimisées autrement appelées \textit{matrices creuses}.
Il aurait peut-être été plus judicieux de stocker une liste d'adjacence directement, mais les matrices creuses sous format Compressed Sparse Row (CSR) peuvent s'apparenter à cela.

Enfin, le sujet demandait à implémenter les plateaux présenté en figure \ref{fig:all-boards}

\begin{figure}
	
\end{figure}
% Quatres plateaux de jeux doivent être implémentés, un carré, un donut, un huit et un trèfle.

\subsubsection{Utilisation de dllib}

Les clients sont des librairies dynamiques chargées avant le début de la partie par le serveur.
Le chargement doit être réalisé avec dllib.